% !Mode:: "TeX:UTF-8"
\cleardoublepage

\chapter{绪论}

\section{研究背景与意义}

制作LaTeX模板的意义主要体现在以下几个方面\citeup{lecun2015deep,grecoNewRoughSet1998}:

% 有序列表的 [fullwidth,itemindent=\parindent] 不可以省略
\begin{enumerate}[fullwidth,itemindent=\parindent]
    \item 提高排版效率:LaTeX是一种基于TEX的排版系统,通过编写代码来实现文档的排版。制作LaTeX模板可以将常用的文档结构、样式和格式进行封装,使作者在撰写新文档时能够快速套用,从而提高排版效率。
    \item 保证文档一致性:LaTeX模板可以帮助确保文档在格式、字体、间距等方面的统一。这在撰写长篇论文、报告或书籍时尤为重要,有助于保持文档的专业性和美观度。
    \item 便于协作与交流:LaTeX模板可以为多人协作提供统一的文档框架和样式。当多个作者共同撰写一篇文档时,使用相同的模板可以避免因格式不一致而产生的问题,提高协作效率。此外,LaTeX模板也有助于学术界的交流,因为许多期刊和会议都提供了官方的LaTeX模板,作者可以直接使用这些模板来撰写论文,确保符合发表要求。
    \item 个性化定制:LaTeX模板可以根据个人或组织的喜好进行定制,包括设置特定的字体、颜色、页眉页脚等。这有助于体现作者或组织的特色,使文档更具个性化。
    \item 促进LaTeX技术的发展:制作LaTeX模板可以推动LaTeX技术的创新和发展。许多优秀的LaTeX宏包和扩展都是在模板制作过程中产生的,这些工具进一步丰富了LaTeX的功能,使其能够应对更多复杂的排版需求。
    \item 传承与积累:LaTeX模板作为一种知识资产,可以在团队和组织内部传承和积累。通过不断优化和更新模板,可以将排版经验和技巧传承给后人,提高整个团队或组织的文档制作水平。
\end{enumerate}

总之,制作LaTeX模板有助于提高排版效率、保证文档一致性、便于协作与交流、个性化定制、促进LaTeX技术的发展以及传承与积累。掌握LaTeX模板制作技巧对于科研人员、工程师、教师等专业人士来说具有重要的实际意义\citeup{online2024deep}。

\section{国内外研究现状}

已完成的内容:

\begin{enumerate}[fullwidth,itemindent=\parindent]
    \item 图表公式编号等
    \item 三线表格式
    \item 等等
\end{enumerate}

\subsection{传统方法}

你可以使用 word。

\subsection{智能方法}

你可以使用 ChatGPT。
\subsubsection{基于深度强化学习的方法}

\section{论文章节安排}
本文分为五章,具体内容安排如下:

第一章,绪论。

第二章,LaTeX 基本使用指南。

第三章,LaTeX 复杂功能使用指南。

第四章,无。

第五章,无。
