% !Mode:: "TeX:UTF-8"

% ======================================= Cover =========================================
\newcommand{\logcolor}{red}                                             % 图标颜色,只有 red 和 black
\newcommand{\titlecn}{湖南师范大学的硕士\\LaTeX毕业论文模板}     % 中文标题
\newcommand{\titleen}{Waiting For Your Thesis\\Hello}          % 英文标题
\newcommand{\titlehead}{湖南师范大学的硕士LaTeX毕业论文模板}      % 页眉标题
\newcommand{\priormajor}{计算机科学与技术}                                       % 学科专业
\newcommand{\authorname}{姓名}                                           % 作者姓名
\newcommand{\supervisor}{导师名字~职称}                                    % 导师
\newcommand{\thesisid}{}                                                % 论文编号
\newcommand{\clcnumber}{}                                               % 分类号
\newcommand{\csecret}{公开}                                             % 密级
\newcommand{\schoolcode}{10542}                                         % 学校代码
\newcommand{\studentid}{202159386276}                                   % 学号
\newcommand{\interestmajor}{深度学习}                                    % 研究方向
\newcommand{\thesisdate}{二零二四年五月}                                 % 论文日期

\makeatletter
%========================================= Font ==========================================
\renewcommand\normalsize{
    \@setfontsize\normalsize{12pt}{12pt}
    \setlength\abovedisplayskip{8pt}
    \setlength\abovedisplayshortskip{8pt}
    \setlength\belowdisplayskip{\abovedisplayskip}
    \setlength\belowdisplayshortskip{\abovedisplayshortskip}
    \let\@listi\@listI
}
\def\defaultfont{\renewcommand{\baselinestretch}{1.5}\normalsize\selectfont}

% 表格字体用 \small 缩小,五号宋体
\renewcommand{\small}{\zihao{5} \song}

% 段落缩进:2 个字符
\setlength{\parindent}{2em}

% 段落间距:涉及目录
% 原代码意思是:段落间距是至少 3pt,但可能会因为页面边距的限制而增加或减少最多 1pt。这样的设置可以使得段落之间有适当的空隙,同时避免了因为页面边距不足而导致段落重叠的情况。
% 默认段间距是 0,符合师大要求,注释这一行
% \setlength{\parskip}{3pt plus1pt minus1pt} % 

% 公式跨页设置,公式之前可以换页,公式出现在页面顶部
\predisplaypenalty=0
\allowdisplaybreaks[4]

% 允许断页时给上一页底部留有一定空白,避免拉大标题和行距
\raggedbottom

%======================================== Header =========================================
% 定义页眉和页脚
\newcommand{\makeheadrule}{\rule[6pt]{\textwidth}{0.5mm} \\[-17.5pt]}
\renewcommand{\headrule}{
    {\if@fancyplain\let\headrulewidth\plainheadrulewidth\fi
     \makeheadrule}}
\pagestyle{fancyplain}

% 不要注销这一行,否则页眉会变成:“第 1 章 1  绪论”样式
\renewcommand{\chaptermark}[1]{\markboth{\chaptertitlename~~\ #1}{}}
\fancyhf{}
\fancyhead[CO]{\zihao{5} \song \titlehead} % 奇数页页眉
\fancyhead[CE]{\zihao{5} \song 硕士学位论文}         % 偶数页页眉
\fancyfoot[C]{\zihao{5} \song \thepage}             % 页码

%=================================== TOC: 目录格式定义 ====================================
\renewcommand\contentsname{目\quad 录}              % 目录名
\setcounter{secnumdepth}{4}                         % 设置章节的编号深度
\setcounter{tocdepth}{2}                            % 设置在 ToC 的显示的章节深度
\dottedcontents{section}[38pt]{}{22pt}{0.3pc}
\dottedcontents{subsection}[70pt]{}{32pt}{0.3pc}

\titlecontents{chapter}[0em]{\vspace{0pt} \zihao{4} \hei}
    {\thecontentslabel~~~ }{}
    {\hspace{0.5em} \titlerule*{.} \contentspage}
\titlecontents{section}[1em]{\vspace{0pt} \zihao{4} \song}
    {\thecontentslabel ~~}{}
    {\hspace{0.5em} \titlerule*{.} \contentspage}
\titlecontents{subsection}[2em]{\vspace{0pt} \zihao{4} \song}
    {\thecontentslabel~~}{}
    {\hspace{0.5em} \titlerule*{.} \contentspage}

%======================================= 标题设置 =========================================

% 标题格式设置
\ctexset{
    % 一级标题用 3 号黑体居中,空一行
    chapter = {
        format = {\linespread{1.0} \zihao{3} \hei \centering},
        name = {第 , 章},
        number = {\arabic{chapter}},
        numberformat = \rmfamily,
        aftername = \quad,
        beforeskip = {7pt},
        afterskip = {18pt},
        pagestyle = plain
    },
    % 二级标题用小 3 号黑体
    section = {
        format = {\linespread{1.0} \zihao{-3} \hei \raggedright},
        numberformat = \rmfamily,
        aftername = \quad,
        beforeskip = {24pt},
        afterskip = {6pt},
    },
    % 三级标题用 4 号黑体
    subsection = {
        format = {\linespread{1.0} \zihao{4} \hei \raggedright},
        numberformat = \rmfamily,
        aftername = \quad,
        beforeskip = {12pt},
        afterskip = {6pt},
    },
    % 四级标题用 4 号楷体
    subsubsection = {
        format = {\linespread{1.0} \zihao{4} \kt \raggedright},
        numberformat = \rmfamily,
        aftername = \quad,
        beforeskip = {12pt},
        afterskip = {6pt},
    }
}

% 附录、摘要放入目录
\newcommand{\Chapter}[1]{
    \phantomsection     % 章节不编号
    \chapter{#1}
    % \markboth{#1}{#1}
}

%===================================== 参考文献设置 =======================================
% 参考文献格式
\bibliographystyle{bib/gbt-7714-2015-numerical}
\addtolength{\bibsep}{-0.8em}
\newcommand{\upcite}[1]{{\setcitestyle{square,super}\citep{#1}}} % 文献上标

%======================================= Theorem =========================================
% 定理正文和标题字体
\theoremstyle{plain}
\theoremheaderfont{\hei\rmfamily}
\theorembodyfont{\rmfamily}
\theoremseparator{:}

% 定理环境和上下文之间的距离
\setlength{\theorempreskipamount}{0pt}
\setlength{\theorempostskipamount}{0pt}

% 定义新的定理环境
\theoremsymbol{} \newtheorem{Definition}{\hskip 2em \hei 定义}[chapter]
\theoremsymbol{} \newtheorem{Proposition}{\hskip 2em \hei 命题}[chapter]
\theoremsymbol{} \newtheorem{Example}{\hskip 2em \hei 例}[chapter]
\theoremsymbol{} \newtheorem{Theorem}{\hskip 2em \hei 定理}[chapter]
\theoremsymbol{} \newtheorem*{Proof}{\hskip 2em \hei 证明}

%========================================= FTE ===========================================
% 图表公式的编号为 1-1 格式,子图编号为 (a) 的格式
%\renewcommand{\figurename}{Figure}% 英文论文图标题以 Figure 开头
%\renewcommand{\tablename}{Table}% 英文论文表标题以 Table 开头
\renewcommand{\thefigure}{\arabic{chapter}-\arabic{figure}}
\renewcommand{\thesubfigure}{(\alph{subfigure})}
\renewcommand{\p@subfigure}{\thefigure~}
\renewcommand{\thetable}{\arabic{chapter}-\arabic{table}}
\renewcommand{\theequation}{\arabic{chapter}.\arabic{equation}}
\renewcommand{\thealgocf}{\thechapter-\arabic{algocf}} 

% 把 tabularx 宏包的列格式左对齐改成居中对齐
\newcolumntype{Y}{>{\centering\arraybackslash}X}

% 定制浮动图形和表格标题样式
\captionnamefont{\zihao{5}}
\captiontitlefont{\zihao{5}}
\captiondelim{~~}
\hangcaption
\setlength{\abovecaptionskip}{4pt}
\setlength{\belowcaptionskip}{0pt}

% 关于公式的几个定义
\renewcommand{\Re}{\mathrm{Re}}
\renewcommand{\Im}{\mathrm{Im}}
\newcommand{\mbf}[1]{\mathbf{#1}}
\newcommand{\Exp}{\mathrm{E}}
\newcommand{\dif}{\mathrm{d}}
\newcommand{\seq}[2]{#1_1,#1_2,\cdots,#1_#2}
\newcommand{\iprod}[2]{\langle #1,#2 \rangle}

% 调整罗列环境的布局
\setitemize{leftmargin=3em,itemsep=0em,partopsep=0em,parsep=0em,topsep=-0em}
\setenumerate{leftmargin=3.2em,itemsep=0em,partopsep=0em,parsep=0em,topsep=0em}

% 自定义项目列表标签及格式 \begin{publist} 列表项 \end{publist}
\newcounter{pubctr} %自定义新计数器
\newenvironment{publist}{\zihao{5}
    \begin{list}{[\arabic{pubctr}]}{
     \usecounter{pubctr}
     \setlength{\leftmargin}{2em}        % 左边界
     \setlength{\labelsep}{0.8em}        % 标号和列表项之间的距离,默认 0.5em
     \setlength{\parsep}{0ex}            % 段落间距
     \setlength{\itemsep}{0ex}            % 标签间距
    }}
{\end{list}}%

% 更改算法标题
\renewcommand{\algorithmcfname}{算法}
% \renewcommand{\algocf@captiontext}[1]{\zihao{4} \song#1}
% \renewcommand\thealgocf{\csname the\algocf@within-\@arabic\c@algocf} % 算法编号
\SetKwInput{KwIn}{输入}
\SetKwInput{KwOut}{输出}
\SetAlCapSty{textrm}
\SetAlCapSkip{1ex}
\normalem % 取消关键字后面的下划线
\makeatother

% 更改代码标题
\renewcommand{\lstlistingname}{代码}
\renewcommand{\thelstlisting}{\arabic{chapter}-\arabic{lstlisting}}

% 上角标形式的参考文献引用格式
\newcommand{\citeup}[1]{\textsuperscript{\cite{#1}}}

%======================================== Other ==========================================
% 避免宏包 hyperref 和 arydshln 不兼容带来的目录链接失效的问题。
\def\temp{\relax}
\let\temp\addcontentsline
\gdef\addcontentsline{\phantomsection\temp}
\gdef\hitempty{}

% 将线宽设置为 1pt
\setlength{\arrayrulewidth}{0.25pt} 

% 保证每一章的首页都在奇数页
% 每一章前面添加 \clearemptydoublepage
\newcommand{\clearemptydoublepage}{%
\clearpage {\pagestyle{empty}
\origdoublepage}
}

% 正文行距设置
\linespread{1.28}