% !Mode:: "TeX:UTF-8"

\usepackage[%
    a4paper,
    top=2.54cm,
    bottom=2.54cm,
    left=3.18cm,
    right=3.18cm,
    includehead,
    includefoot
]{geometry}                 % 控制页面尺寸
\usepackage{titletoc}       % 控制目录的宏包
\usepackage{titlesec}       % 控制标题的宏包, 缩进
\usepackage{fancyhdr}       % 页眉和页脚的相关定义
\usepackage{color}          % 支持彩色
\usepackage{graphicx}       % 处理图片
\usepackage{float}          % 浮动体处理
\usepackage{tabularx}       % 可伸缩表格
\usepackage{multirow}       % 表格可以合并多个row
\usepackage{booktabs}       % 表格横的粗线;\specialrule{1pt}{0pt}{0pt}
\usepackage{longtable}      % 支持跨页的表格。
\usepackage{enumitem}       % 使用enumitem宏包,改变列表项的格式
\usepackage{amsmath}        % 公式宏包
\usepackage{amssymb}        % 符号宏包
\usepackage{amsfonts}       % 大写空心粗体字、欧拉字体宏包
\usepackage{ulem}           % 线宏包
\usepackage{bm}             % 处理数学公式中的黑斜体的宏包
\usepackage{lmodern}        % 数学公式字体
\usepackage{etoolbox}       % 盒子
\usepackage{wasysym}        % Check Box Marks
\usepackage{setspace}       % 行间距的宏包
\usepackage[amsmath,thmmarks,hyperref]{ntheorem}    % 定理类环境宏包
\usepackage[hang]{subfigure}                        % 图形或表格并排排列
\usepackage[subfigure]{ccaption}                    % 支持双语标题
\usepackage[sort&compress,numbers]{natbib}            % 支持引用缩写的宏包
\usepackage{fontspec}        % 字体设置宏包

% 生成有书签的 pdf 及其开关, 该宏包应放在所有宏包的最后, 宏包之间有冲突
\usepackage[%
    xetex,
    bookmarksnumbered=true,
    bookmarksopen=true,
    colorlinks=false,
    pdfborder={0 0 0.7},
    citebordercolor={0 0 1},
    breaklinks=true
]{hyperref}
\hypersetup{hidelinks}        % hide links (remove color and border).

% 算法的宏包,注意宏包兼容性,先后顺序为float、hyperref、algorithm(2e),否则无法生成算法列表
% linesnumbered         显示行号
% ruled                 标题显示在上方,不加就默认显示在下方
% vlined                代码段中用线连接
% boxed                 将算法插入在一个盒子里
% algochapter           算法是否在章节编号中编号
\usepackage[linesnumbered,ruled]{algorithm2e}

% 为了避免与页眉的兼容问题可将listings放入table环境中
\usepackage{listings}
\definecolor{codegreen}{rgb}{0,0.6,0}
\definecolor{codepurple}{rgb}{0.58,0,0.82}
\lstset{%
    language={[ISO]C++},                % 设置语言
    alsolanguage=Matlab,                % 可以添加多个语言选项
    alsolanguage=Verilog,
    morekeywords={numerictype,fimath,fipref,fi,trh},
    commentstyle=\color{codegreen},        % 注释颜色
    keywordstyle=\color{blue},            % 代码关键字颜色
    stringstyle=\color{codepurple},        % 代码中字符串颜色
    frame=single,            % 设置边框, 
    xleftmargin=1.7em,        % 设定listing左边空白 
    numbers=left,            % 左侧显示行号
    numberstyle=\small,        % 行号字体用小号
    breaklines=true,        % 对过长的代码自动换行 
    columns=flexible,        % 调整字符之间的距离
    tabsize=4                % table 长度
}

 \usepackage{nomencl} % 使用常用符号表,需要用到指令Makeindex指令生成符号表
 \makenomenclature

% 带圆圈的脚注
\usepackage{pifont}
\usepackage[flushmargin,para,symbol*]{footmisc}
\DefineFNsymbols{circled}{{\ding{192}}{\ding{193}}{\ding{194}}
    {\ding{195}}{\ding{196}}{\ding{197}}{\ding{198}}{\ding{199}}{\ding{200}}{\ding{201}}}
\setfnsymbol{circled}

% 字体设置,Windows自带黑体(simhei.ttf)和宋体(simsun.ttf)
\defaultfontfeatures{Mapping=tex-text}
\setmainfont{Times New Roman}
\setsansfont{Arial}
\setCJKmainfont{SimSun}
\setCJKsansfont{SimHei}
\setCJKfamilyfont{hei}{SimHei}
\newcommand{\hei}{\CJKfamily{hei}}      % 黑体
\setCJKfamilyfont{song}{SimSun}
\newcommand{\song}{\CJKfamily{song}}    % 宋体
\setCJKfamilyfont{fs}{FangSong}
\newcommand{\fs}{\CJKfamily{fs}}        % 仿宋
\setCJKfamilyfont{kt}{KaiTi}
\newcommand{\kt}{\CJKfamily{kt}}        % 楷体
